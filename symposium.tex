\section{The Graph Drawing Symposium and Contest}
For over 25 years, an international symposium of Graph Drawing and Network Visualization takes place annually. \textcolor{red}{Wo überall bisher?} In the year 2021, the $28^{\text{th}}$ International Symposium of Graph Drawing and Network Visualization will be held from September $15^{\text{th}}$ to $17^{\text{th}}$ in Tübingen. 
\newline Part of the symposium is a traditional Graph Drawing Contest. The contest consists of two parts - the \textit{Creative Topics} and the \textit{Live Challenge}. The main focus for the Creative Topics lies on the creation of drawings of two given graphs. Aspects to consider for the visualization are clarity, aesthetic appeal and readability.
\newline On the other hand, the Live Challenge is held similar to a programming contest. Participants, usually teams, will get a theme and a set of graphs and will have one hour of processing. The results will be ranked and the team with the highest score wins the competition. The teams will be allowed to use any combination of software and human interaction systems in order to produce the best results. Usually, the challenge is derived from a theoretical optimization problem.
\newline This year, there will be two categories that are judged independently:
\begin{description}
	\item[Automatic:] The teams will use their own defined toolchain. Therefore, the challenge graphs will be large ones.
	\item[Manual:] A given graph tool will only provide the  functionality to manually move objects. This prohibits any algorithm to work on the solution.
\end{description}
\subsection{The challenge}
There has been recent attention to the edge-length ratio of a planar drawing, which describes the ratio between the lengths of the longest edge and the shortest edge in a drawing. 
\newline This year, the main topic addresses an optimization problem, namely the minimization of the edge-length ratio of poly-line drawings of planar, undirected graphs on a fixed grid. For a poly-line edge, the edge-length is the sum of the line segment lengths.
\bigbreak The input consists of a JSON file with the following entries:
\begin{description}
	\item[nodes] Every node has an unique ID value between 0 and the amount of nodes - 1, a value for the $x$ and $y$ coordinate each, delimited by the width and height
	\item[edges] Every edge has an ID for source and destination each and an optional list of bend points, specified in $x$ and $y$ coordinate
	\item[width (optional)] The maximum $x$-coordinate of the grid. If unspecified, the width is set to 1,000,000.
	\item[height (optional)] The maximum $y$ coordinate of the grid. If unspecified, the height is set to 1,000,000.
	\item[bends] The maximum number of bends allowed per edge
\end{description}
The results of the optimization are also JSON files. The planarity of the graph shall be preserved and the poly-line edge-length ratio minimized by relocation of the nodes.
\bigbreak For the teams participating with their own tools, an embedding might not be given with the input. For the participants working manually, an embedding is already given beforehand.
