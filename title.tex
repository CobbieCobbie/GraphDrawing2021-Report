% das Papierformat zuerst
%\documentclass[a4paper, 11pt]{article}

% deutsche Silbentrennung
%\usepackage[ngerman]{babel}

% wegen deutschen Umlauten
%\usepackage[ansinew]{inputenc}

% hier beginnt das Dokument
%\begin{document}


\thispagestyle{empty}

%\begin{figure}[t]
% \centering
% \includegraphics[width=0.6\textwidth]{abb/logo1}
%~~~~~~~~~~
% \includegraphics[width=0.3\textwidth]{abb/logo2}
%\end{figure}


\begin{verbatim}
	
	
\end{verbatim}

\begin{center}
	\Large{Eberhard Karls Universität Tübingen}\\
	\small Wilhelm Schickard Institut Tübingen\\
\end{center}


\begin{center}
	\Large{Fachbereich Informatik}
\end{center}
\begin{verbatim}
	
	
	
	
\end{verbatim}
\begin{center}
	%\doublespacing
	\textbf{\LARGE{Minimizing the edge length ratio of planar poly-line graph drawings}}\\
	%\singlespacing
	\begin{verbatim}
		
	\end{verbatim}
	\textbf{{Arbeitsbereich Algorithmik}}
\end{center}
\begin{verbatim}
	
\end{verbatim}
\begin{center}
	
\end{center}
\begin{verbatim}
	
\end{verbatim}
\begin{center}
	\textbf{Forschungsprojekt SS21}
\end{center}
\begin{verbatim}
	
	
	
	
	
	
\end{verbatim}
\begin{flushleft}
	\begin{tabular}{llll}
		\textbf{Autor:} & & Benjamin Ulvi \c Coban & \\
		& & MatNr. 3526251 & \\
		& & \\
		\textbf{Version vom:} & & \today &\\
		& & \\
		\textbf{Betreuer:} & & Prof. Dr. Michael Kaufmann &\\
	\end{tabular}
\end{flushleft}