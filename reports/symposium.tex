\section{The Graph Drawing Symposium and Contest}
For over 25 years, an international symposium of Graph Drawing and Network Visualization takes place annually. \textcolor{red}{Wo überall bisher?} In the year 2021, the $28^{\text{th}}$ International Symposium of Graph Drawing and Network Visualization will be held from September $15^{\text{th}}$ to $17^{\text{th}}$ in Tübingen. 
\newline Part of the symposium is a traditional Graph Drawing Contest. The contest consists of two parts - the \textit{Creative Topics} and the \textit{Live Challenge}. The main focus for the Creative Topics lies on the creation of drawings of two given graphs. Aspects to consider for the visualization are clarity, aesthetic appeal and readability.
\newline On the other hand, the Live Challenge is held similar to a programming contest. Participants, usually teams, will get a theme and a set of graphs and will have one hour of processing. The results will be ranked and the team with the highest score wins the competition. The teams will be allowed to use any combination of software and human interaction systems in order to produce the best results. Usually, the challenge is derived from a theoretical optimization problem.
\newline This year, there will be two categories that are judged independently:
\begin{description}
	\item[Automatic:] The teams will use their own defined toolchain. Therefore, the challenge graphs will be large ones.
	\item[Manual:] A given graph tool will only provide the  functionality to manually move objects. This prohibits any algorithm to work on the solution.
\end{description}