\section{Preliminaries}
As otherwise mentioned, a \textit{graph} $G=(V_G,E_G)$ is a tuple consisting of two sets - the set of vertices and the set of edges. An \textit{edge} $e = (v,w), v,w \in V_G$ is a tuple and describes a connectivity relation between two vertices. Unless otherwise mentioned, the graphs are \textit{undirected}. It means that the edge $(u,v)$ is identical to the edge $(v,u)$,$u,v\in V_G$. A \textit{face} is a maximal open region of the plane bounded by edges. The degree of the graph is the amount of edges attached to it. A graph $G'$ is called a supergraph of $G$ iff $V_{G}\subseteq V_{G'}$ and $E_{G}\subseteq E_{G'}$. 
\bigskip\\
 A \textit{drawing} $\Gamma$ of a graph $G$ is a function, where each vertex is mapped on a unique point $\Gamma(v)$ in the plane and each edge is mapped on an open Jordan curve $\Gamma(e)$ ending in its vertices. A graph is \textit{planar} if and only if there exists a crossing-free representation in the plane. $G$ is maximal planar iff any further edge insertion violates the planarity property. A $k\times k$ grid is an undirected graph consisting of $k$ rows and $k$ columns of vertices. A vertex in the $i$-th row and $j$-th column is denoted as $(i,j)$.\\
 A straight line drawing on a grid of size $k\times k$ is a drawing where every vertex has its unique row and column value and every edge is drawn as a straight line. In a polyline drawing, each edge is represented by a non-empty sequence of line segments ($e = (e_1,e_2,...)$), where two consecutive line segments intersect in a unique point, a bend. Every bend lies, like the vertices, on a unique grid point.\bigskip\\
 To measure the length, the euclidian distance of a line segment is introduced as a metric. It is defined as the square root of the sum of the quadratic difference of row and column. The unit length, \UL in short, is defined as the distance between two consecutive points on the grid with either the same row or column value.\bigskip\\
 A path is a sequence of edges. A cycle is a path, so that the starting vertex and the ending vertex are identical. A tree is an acyclic graph. A graph is called connected if there is a path from every vertex to every other vertex of $G$. A $k$-ary tree is a graph where either every vertex has exactly $k$ children and one parent or is of degree 1. These are called leaves. The root of a tree is a vertex with no parent. The height of a tree is defined as the length of the longest path starting from the root. A tree is called complete, when all leaves have the same height and no further vertices can be inserted without increasing the maximum height.\bigskip\\
 A 2-terminal series-parallel graph with terminals $s,t$ is a recursively defined graph with one of the following three rules:
 \begin{enumerate}
 	\item An edge $(s,t)$ is a 2-terminal series-parallel graph
 	\item If $G_i, i = 1,2$, is a 2-terminal series-parallel graph with terminals $s_i,t_i$, then in the serial composition $t_1$ is identified with $s_2$ to obtain a 2-terminal series-parallel graph with $s_1,t_2$ as terminals
 	\item If $G_i, i=1,...,k$, is a 2-terminal series-parallel graph with terminals $s_i,t_i$, then in a parallel composition we identify all $s_i$ into one terminal $s$ and all $t_i$ into the other terminal $t$ and the result is a 2-terminal series-parallel graph with terminals $s,t$.
 \end{enumerate}
A series-parallel graph, SP-graph in short, is a graph for which every biconnected component is a 2-terminal series-parallel graph. A SP-graph is maximal if no edge can be added so while maintaining a SP-graph.\bigskip\\
A 2-tree is a recursively defined graph with at least three vertices. If $n = 3$, then the 2-tree is the $K_3$. If $n>3$, then start with a $K_3$ and every vertex added is adjacent to exactly two adjacend neighbours, forming a 3-clique. The class of 2-trees correspond to the class of maximal SP-graphs.~~[\cite{DBLP:journals/dcg/Biedl11}, Preliminaries], [\cite{DBLP:journals/jgaa/MondalNRA11}, Preliminaries], [\cite{DBLP:books/daglib/0023376}]
